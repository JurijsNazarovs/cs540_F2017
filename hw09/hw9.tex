\documentclass{article}

\usepackage[top=2in, bottom=1.5in, left=1in, right=1in]{geometry}
\usepackage{fancyhdr} 		% page header
\usepackage{amsmath}
\usepackage{amssymb}
\usepackage{float}
\usepackage{graphicx}
\usepackage{url}
\usepackage{color}
\usepackage{tikz}
\usepackage{hyperref}

\pagestyle{fancy} 
\fancyhf{}
\fancyhead[L]{CS 540}
\fancyhead[R]{Fall 2017}

\def\x{\mathbf{x}}
\def\y{\mathbf{y}}
\def\w{\mathbf{w}}
\def\p{\mathbf{p}}

\begin{document}

\begin{center}
{\bf \large CS 540: Introduction to Artificial Intelligence

Homework Assignment \# 9

\vspace{0.5cm}

Assigned:  11/20 

Due:  12/1 before class} 
\end{center}

\vspace{1cm}

\begin{center}
{\bf \Large Hand in your homework:}
\end{center}

If a homework has programming questions, please hand in the Java program.  
If a homework has written questions, please hand in a PDF file.
Regardless, please zip all your files into hwX.zip where X is the homework number.
Go to UW Canvas, choose your CS540 course, choose Assignment, click on Homework X: this is where you submit your zip file. 

\vspace{1cm}

\begin{center}
{\bf \Large Late Policy:}
\end{center}

All assignments are due at the beginning of class on the due date. One (1) day late, defined as a 24-hour period from the deadline (weekday or weekend), will result in 10\% of the  total points for the assignment deducted.  So, for example, if a 100-point assignment is due on a Wednesday 9:30 a.m., and it is handed in between Wednesday 9:30 a.m. and Thursday 9:30 a.m., 10 points will be deducted. Two (2) days late, 25\% off; three (3) days late, 50\% off. No homework can be turned in more than three (3) days late. Written questions and program submission have the same deadline.  

Assignment grading questions must be raised with the instructor within one week after the assignment is returned.

\vspace{4pt}

\begin{center}
{\bf \Large Collaboration Policy:}
\end{center}

You  are  to  complete  this  assignment  individually.  However,  you  are  encouraged  to  discuss  the  general algorithms and ideas with classmates, TAs, and instructor in order to help you answer the questions. You are also welcome to give each other examples that are not on the assignment in order to demonstrate how to solve problems. But we require you to:
\begin{itemize}
\item not explicitly tell each other the answers
\item not to copy answers or code fragments from anyone or anywhere
\item not to allow your answers to be copied
\item not to get any code on the Web
\end{itemize}

In those cases where you work with one or more other people on the general discussion of the assignment and surrounding topics, we suggest that you 
specifically record on the assignment the names of the people you were in discussion with.

\newpage



\section*{Question 1: Lake Mendota Ice [100 points]}

Can you predict how harsh this Wisconsin winter will be?

The Wisconsin State Climatology Office keeps a record on
the number of days Lake Mendota was covered by ice at
\url{http://www.aos.wisc.edu/~sco/lakes/Mendota-ice.html}.

Write a program \textbf{Ice.java} with the following command line format:
\begin{verbatim}
$java Ice FLAG [arg1 arg2]
\end{verbatim}
Where the two optional arguments are real valued.

Questions 5 is 20 points; other questions 10 points each.

\begin{enumerate}
\item
As with any real problems, the data is not as clean or as organized as one would like for machine learning.
Curate a clean data set starting from 1855-56 and ending in 2016-17.
Let $x$ be the year: for 1855-56, $x=1855$; for 2016-17, $x=2016$; and so on.
Let $y$ be the ice days in that year: for 1855-56, $y=118$; for 2016-17, $y=65$; and so on.
Some years have multiple freeze thaw cycles such as 2001-02, that one should be $x=2001, y=21$.
Although we do not ask you to hand in any visualization code or figures, we strongly advise you to plot the data (using any plotting tool inside or outside Java) and see what it is like.

For simplicity, hard code the data set in your program.
When FLAG=100, print out the data set.  One year per line with the $x$ value first, a space, then the $y$ value.
For example,
\begin{verbatim}
$java Ice 100
1855 118
...
2001 21
...
2016 65
\end{verbatim}

\item
When FLAG=200, print $n$ the number of data points,  the sample mean $\bar y=\frac{1}{n}\sum_{i=1}^n y_i$, and the sample standard deviation $\sqrt{\frac{1}{n-1}\sum_{i=1}^n (y_i-\bar y)^2}$, on three lines.  
For real values in this homework, keep two digits after decimal point.
For example (the numbers are made-up):
\begin{verbatim}
$java Ice 200
162
123.45
32.10
\end{verbatim}

\item
We will perform linear regression with the model
$$f(x) = \beta_0 + \beta_1 x.$$
We first define the mean squared error as a function of $\beta_0, \beta_1$:
$$MSE(\beta_0, \beta_1) = \frac{1}{n} \sum_{i=1}^n (\beta_0 + \beta_1 x_i - y_i)^2.$$
When FLAG=300, arg1=$\beta_0$ and arg2=$\beta_1$.
Print the corresponding MSE.
\begin{verbatim}
$java Ice 300 0 0
10897.85
$java Ice 300 100.00 0
386.74
$java Ice 300 300.00 -0.10
333.08
$java Ice 300 400 0.1
241561.79
$java Ice 300 200 -0.2 
84199.70
\end{verbatim}

\item
We perform gradient descent on MSE.  
At current parameter $(\beta_0, \beta_1)$, the gradient is defined by the vector of partial derivatives
\begin{eqnarray}
\frac{\partial MSE(\beta_0, \beta_1)}{\partial \beta_0} &=& \frac{2}{n}\sum_{i=1}^n (\beta_0 + \beta_1 x_i - y_i) \\
\frac{\partial MSE(\beta_0, \beta_1)}{\partial \beta_1} &=& \frac{2}{n}\sum_{i=1}^n (\beta_0 + \beta_1 x_i - y_i) x_i.
\end{eqnarray}
When FLAG=400, arg1=$\beta_0$ and arg2=$\beta_1$.
Print the corresponding gradient as two numbers on separate lines.
\begin{verbatim}
$java Ice 400 0 0
-205.11
-396150.93
$java Ice 400 100 0
-5.11
-9050.93
$java Ice 400 300 -0.1
7.79
15479.64
$java Ice 400 400 0.1
981.99
1901918.51
$java Ice 400 200 -0.2
-579.31
-1121289.79
\end{verbatim}

\item
Gradient descent starts from initial parameter $(\beta_0^{(0)}, \beta_1^{(0)})$,
and iterates the following updates at time $t=1,2,\ldots,T$:
\begin{eqnarray}
\beta_0^{(t)} &=& \beta_0^{(t-1)} - \eta \frac{\partial MSE(\beta_0^{(t-1)}, \beta_1^{(t-1)})}{\partial \beta_0} \\
\beta_1^{(t)} &=& \beta_1^{(t-1)} - \eta \frac{\partial MSE(\beta_0^{(t-1)}, \beta_1^{(t-1)})}{\partial \beta_1}.
\end{eqnarray}
When FLAG=500, arg1=$\eta$ and arg2=$T$.
Start from initial parameter $(\beta_0^{(0)}, \beta_1^{(0)})=(0,0)$.
Perform $T$ iterations of gradient descent.
Print the following in each iteration on a line, separated by space:
$t, \beta_0^{(t)}, \beta_1^{(t)}, MSE(\beta_0^{(t)}, \beta_1^{(t)})$.
For example,
\begin{verbatim}
$java Ice 500 1e-7 5
1 0.00 0.04 1086.78
2 0.00 0.05 471.97
3 0.00 0.05 433.44
4 0.00 0.05 431.03
5 0.00 0.05 430.88

$java Ice 500 1e-8 5
1 0.00 0.00 9387.32
2 0.00 0.01 8094.78
3 0.00 0.01 6988.77
4 0.00 0.01 6042.37
5 0.00 0.02 5232.56

$java Ice 500 1e-9 5
1 0.00 0.00 10741.50
2 0.00 0.00 10587.49
3 0.00 0.00 10435.78
4 0.00 0.00 10286.34
5 0.00 0.00 10139.12

$java Ice 500 1e-6 5
1 0.00 0.40 442211.52
2 -0.00 -2.18 18646694.58
3 0.01 14.54 787004462.48
4 -0.05 -94.08 33217129357.42
5 0.32 611.63 1401997461432.22
\end{verbatim}
Note with $\eta=1e-6$ gradient descent is diverging.

The following is not required for hand in, but try different initial parameters, $\eta$, and much larger $T$ and see how small you can make MSE.

\item
Instead of using gradient descent, we can compute the closed-form solution for the parameters directly.
For ordinary least squared in 1D, this is
$$\hat\beta_1 = \frac{\sum (x_i - \bar x) (y_i - \bar y)}{\sum (x_i - \bar x)^2}$$
$$\hat\beta_0 = \bar y - \hat\beta_1\bar x,$$
where $\bar x = (\sum x_i)/n$
and $\bar y = (\sum y_i)/n$.
Required: When FLAG=600, print $\hat\beta_0, \hat\beta_1$ (note the order), and the corresponding MSE on a single line separated by space.

The following is not required for hand in, but give it some thought: what does a negative $\hat\beta_1$ mean?


\item
Use $\hat\beta_0, \hat\beta_1$ you can predict the number of ice days for a future year.
Required: When FLAG=700, arg1=year.  Print a single real number which is the predicted ice days for that year.
For example,
\begin{verbatim}
$java Ice 700 2020
86.30
\end{verbatim}

The following is not required for hand in, but give it some thought:
\begin{itemize}
\item What's the prediction for this winter?
\item Which year will the predicted ice day become negative?
\item What does that say about the model?
\end{itemize}

\item
If you are agonizing over your inability to get gradient descent to match the closed-form solution in question 5, you are not alone. 
The culprit is the scale of input $x$ compared to the scale of the implicit offset value 1 (think $\beta_0 = \beta_0 \cdot 1$).
Gradient descent converges slowly when these scales differ greatly, a situation known as bad condition number in optimization.
When FLAG=800, we ask you to first normalize the input $x$ (note: not the labels $y$):
\begin{eqnarray}
\bar x &=& \frac{1}{n}\sum_{i=1}^n x_i \\
std_x &=& \sqrt{\frac{1}{n-1}\sum_{i=1}^n (x_i - \bar x)^2 } \\
x_i &\leftarrow& (x_i - \bar x)/std_x.
\end{eqnarray}
Then proceed exactly as when FLAG=500.  The two arguments are again arg1=$\eta$ and arg2=$T$.
For example,
\begin{verbatim}
$java Ice 800 0.1 5
1 20.51 -1.79 7082.51
2 36.92 -3.23 4640.63
3 50.05 -4.38 3077.80
4 60.55 -5.31 2077.56
5 68.95 -6.05 1437.39

$java Ice 800 1 5
1 205.11 -17.94 10895.86
2 0.00 -0.22 10893.93
3 205.11 -17.72 10892.04
4 -0.00 -0.44 10890.19
5 205.11 -17.51 10888.39

$java Ice 800 0.01 5
1 2.05 -0.18 10478.17
2 4.06 -0.36 10075.10
3 6.03 -0.53 9687.99
4 7.96 -0.70 9316.22
5 9.85 -0.86 8959.16
\end{verbatim}
With $\eta=0.1$ you should get convergence within 100 iterations.

(Note the $\beta$'s are now for the normalized version of $x$, but you can easily translate them back for the original $x$ with a little algebra.  This is not required for the homework.)

\item
Now we implement Stochastic Gradient Descent (SGD).
With everything the same as part 8 (including $x$ normalization), we modify the definition of gradient in equations (1) and (2) as follows.
In iteration $t$ we randomly pick one of the $n$ items.  Say we picked $(x_{j_t}, y_{j_t})$.
We approximate the gradient using that item only:
\begin{eqnarray}
\frac{\partial MSE(\beta_0, \beta_1)}{\partial \beta_0} &\approx& 2( \beta_0 + \beta_1 x_{j_t} - y_{j_t}) \\
\frac{\partial MSE(\beta_0, \beta_1)}{\partial \beta_1} &\approx& 2 (\beta_0 + \beta_1 x_{j_t} - y_{j_t}) x_{j_t}.
\end{eqnarray}
When FLAG=900, print the same information as in part 8.  For example (your results will differ because of randomness in the items picked):
\begin{verbatim}
$java Ice 900 0.1 5
1 24.60 -36.45 7123.53
2 42.85 -40.53 4850.61
3 49.01 -47.42 4631.96
4 45.74 -42.09 4614.13
5 67.18 -8.49 1550.91
\end{verbatim}
With $\eta=0.1$ you should approximately converge within a few hundred iterations.

Since $n$ is small in our data set, there is little advantage of SGD over gradient descent.  However, on large data sets SGD becomes more desirable.

\end{enumerate}

\end{document}
